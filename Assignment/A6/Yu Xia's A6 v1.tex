% Name: Yu Xia's A6 v1
% Author: Yu Xia 
% Description: The version before submitted (to be cont.). Correct spelling and grammar.
% Last Updated: Nov 23, 2022

\documentclass{article}

\usepackage{titlesec}

\usepackage{amsmath}
\usepackage{amssymb}

\usepackage{booktabs}
\usepackage{float}
\usepackage{colortbl}
\usepackage{xcolor}

\usepackage{a4wide}
\usepackage{setspace}
\usepackage{geometry}
\usepackage{parskip}

\usepackage{multirow}
\usepackage{adjustbox}
\usepackage{graphicx}

\usepackage{listings}
\lstdefinestyle{mystyle}{
    basicstyle=\ttfamily
}

\lstset{style=mystyle
}

\renewcommand{\familydefault}{\sfdefault}

\usepackage{hyperref}
\hypersetup{
    colorlinks=true,
    linkcolor=black,
    urlcolor=blue
}

\DeclareRobustCommand{\bbone}{\text{\usefont{U}{bbold}{m}{n}1}}

\titleformat*{\section}{\Large\rmfamily\bfseries}

\titleformat*{\subsection}{\large\rmfamily\bfseries}

\titleformat*{\subsubsection}{\normalfont\rmfamily}

\titleformat*{\paragraph}{\normalfont\rmfamily}

\author{Yu Xia \\ ID: yx5262}
\title{\textbf{Yu Xia's A6}}
\date{Fall 2022}

\begin{document}
\maketitle

\nocite{*}

\section*{Multiple Choice}

1-5:\\
cdeac

6-10:\\
badac

11-15:\\
bbcea

16-20:\\
adacb

\section*{Long}

\subsection*{1. Labor Search Model A}

\subsubsection*{a.}

Consumer's job-finding constraint:

\fbox{%
	\parbox{1\linewidth}{%
\begin{flalign*} \label{eq:1.1}
    \left(1-\rho\right)n^{s}_{t-1}+p^{f}_{t}s_{t}&=n^{s}_{t}& \tag{1.1}
\end{flalign*}%
}%
}

Consumer's labor is determined by the proportion of people keeping their job, the probability of finding a job, and the search effort $s_{t}$. It is different from the classical framework which focuses on the choice of labor and consumption, without probability.

Consumer's budget constraint:

\fbox{%
	\parbox{1\linewidth}{%
\begin{flalign*} \label{eq:1.2}
    c_{t}&=\left(1-t_{t}\right)w_{t}n^{s}_{t}+\left(1-p^{f}_{t}\right)s_{t}b^{ue}_{t}& \tag{1.2}
\end{flalign*}%
}%
}

When consumers find a job (in other words, those people who have a job), they get the salary according to their labor, otherwise (unemployment pool) they spend the time to find a job and get the unemployment benefit. It can also be expressed as the expectation: the case of having a job and getting paid, and the case of finding a job and getting unemployment benefits.

The difference between the dynamic model and the static model is that when we introduce $\rho$, there are workers who stay in their job for the next period. Variables could be different in each time period, so we have the subscript.

\subsubsection*{b.}

\fbox{%
	\parbox{1\linewidth}{%
The Lagrangian:
\begin{align*}
    \quad\mathcal{L} & = \sum_{t=0}^{\infty}\beta^{t}\Biggl(\left(u\left(c_{t}\right)-h\left(\left(1-p^{f}_{t}\right)s_{t}+n^{s}_{t}\right)\right)& \\
    &\quad\quad-\lambda^{h}_{t}\left(c_{t}-\left(1-t_{t}\right)w_{t}n^{s}_{t}-\left(1-p^{f}_{t}\right)s_{t}b^{ue}_{t}\right)&\\
    &\quad\quad-\mu^{h}_{t}\left(\left(1-\rho\right)n^{s}_{t-1}+p^{f}_{t}s_{t}-n^{s}_{t}\right)\Biggr)&\\
    &=\dots&\\
    &\quad+u\left(c_{t}\right)-h\left(\left(1-p^{f}_{t}\right)s_{t}+n^{s}_{t}\right)& \\
    &\quad\quad-\lambda^{h}_{t}\left(c_{t}-\left(1-t_{t}\right)w_{t}n^{s}_{t}-\left(1-p^{f}_{t}\right)s_{t}b^{ue}_{t}\right)&\\
    &\quad\quad-\mu^{h}_{t}\left(\left(1-\rho\right)n^{s}_{t-1}+p^{f}_{t}s_{t}-n^{s}_{t}\right)&\\
    &\quad+\beta u\left(c_{t+1}\right)-h\left(\left(1-p^{f}_{t+1}\right)s_{t+1}+n^{s}_{t+1}\right) &\\
    &\quad\quad-\beta\lambda^{h}_{t+1}\left(c_{t+1}-\left(1-t_{t+1}\right)w_{t+1}n^{s}_{t+1}-\left(1-p^{f}_{t+1}\right)s_{t+1}b^{ue}_{t+1}\right)&\\
    &\quad\quad-\beta\mu^{h}_{t+1}\left(\left(1-\rho\right)n^{s}_{t}+p^{f}_{t+1}s_{t+1}-n^{s}_{t+1}\right)&\\
    &\quad+\dots
\end{align*}%
}%
}

The choice variables: $\boxed{c_{t}$, $n^{s}_{t}$, $s_{t}}$ (in each period).

The given variables: $\boxed{p^{f}_{t}$, $w_{t}$, $b^{ue}_{t}$, $t_{t}$, $\rho$, $\beta}$.

\subsubsection*{c.}

\fbox{%
	\parbox{1\linewidth}{%
    FOCs:

    w.r.t $c_{t}$:
    \begin{flalign} \label{eq:1.3}
        u^{\prime}-\lambda^{h}_{t}&=0& \tag{1.3}
    \end{flalign}

    w.r.t $n^{s}_{t}$:
    \begin{flalign*} \label{eq:1.4}
        -h^{\prime}\left(\left(1-p^{f}_{t}\right)s_{t}+n^{s}_{t}\right)+\lambda^{h}_{t}\left(1-t_{t}\right)w_{t}+\mu^{h}_{t}-\beta\mu^{h}_{t+1}\left(1-\rho\right)&=0& \tag{1.4}
    \end{flalign*}

    w.r.t $s_{t}$:
    \begin{flalign*} \label{eq:1.5}
        -h^{\prime}\left(\left(1-p^{f}_{t}\right)s_{t}+n^{s}_{t}\right)\left(1-p^{f}_{t}\right)+\lambda^{h}_{t}\left(1-p^{f}_{t}\right)b^{ue}_{t}-\mu^{h}_{t}p^{f}_{t}&=0& \tag{1.5}
    \end{flalign*}%
}%
}

\eqref{eq:1.4} $\iff$\begin{flalign*} 
    h^{\prime}\left(\left(1-p^{f}_{t}\right)s_{t}+n^{s}_{t}\right)&=u^{\prime}\left(c_{t}\right)\left(1-t_{t}\right)w_{t}+\mu^{h}_{t}-\beta\mu^{h}_{t+1}\left(1-\rho\right)& 
\end{flalign*}

\eqref{eq:1.5} $\iff$\begin{flalign*} 
    h^{\prime}\left(\left(1-p^{f}_{t}\right)s_{t}+n^{s}_{t}\right)\left(1-p^{f}_{t}\right)&=u^{\prime}\left(c_{t}\right)\left(1-p^{f}_{t}\right)b^{ue}_{t}-\mu^{h}_{t}p^{f}_{t}&
\end{flalign*}

$\iff$
\begin{flalign*} 
    h^{\prime}\left(\left(1-p^{f}_{t}\right)s_{t}+n^{s}_{t}\right)\left(1-p^{f}_{t}\right)-u^{\prime}\left(c_{t}\right)\left(1-p^{f}_{t}\right)b^{ue}_{t}&=-\mu^{h}_{t}p^{f}_{t}&
\end{flalign*}

$\iff$
\begin{flalign*} 
    u^{\prime}\left(c_{t}\right)\left(1-p^{f}_{t}\right)b^{ue}_{t}-h^{\prime}\left(\left(1-p^{f}_{t}\right)s_{t}+n^{s}_{t}\right)\left(1-p^{f}_{t}\right)&=\mu^{h}_{t}p^{f}_{t}&
\end{flalign*}

$\iff$
\begin{flalign*} 
    \dfrac{u^{\prime}\left(c_{t}\right)\left(1-p^{f}_{t}\right)b^{ue}_{t}-h^{\prime}\left(\left(1-p^{f}_{t}\right)s_{t}+n^{s}_{t}\right)\left(1-p^{f}_{t}\right)}{p^{f}_{t}}&=\mu^{h}_{t}&
\end{flalign*}

$\iff$
\begin{flalign*} 
    \left(u^{\prime}\left(c_{t}\right)b^{ue}_{t}-h^{\prime}\left(\left(1-p^{f}_{t}\right)s_{t}+n^{s}_{t}\right)\right)\dfrac{1-p^{f}_{t}}{p^{f}_{t}}&=\mu^{h}_{t}&
\end{flalign*}

And
\begin{flalign*} 
    \left(u^{\prime}\left(c_{t+1}\right)b^{ue}_{t+1}-h^{\prime}\left(\left(1-p^{f}_{t+1}\right)s_{t+1}+n^{s}_{t+1}\right)\right)\dfrac{1-p^{f}_{t+1}}{p^{f}_{t+1}}&=\mu^{h}_{t+1}&
\end{flalign*}

\begin{flalign*} 
    \therefore h^{\prime}\left(lfp_{t}\right)&=u^{\prime}\left(c_{t}\right)\left(1-t_{t}\right)w_{t}+\left(u^{\prime}\left(c_{t}\right)b^{ue}_{t}-h^{\prime}\left(lfp_{t}\right)\right)\dfrac{1-p^{f}_{t}}{p^{f}_{t}}&\\
    &\quad\quad -\beta\left(u^{\prime}\left(c_{t+1}\right)b^{ue}_{t+1}-h^{\prime}\left(lfp_{t+1}\right)\right)\dfrac{1-p^{f}_{t+1}}{p^{f}_{t+1}}\left(1-\rho\right)&
\end{flalign*}

$\iff$
\begin{flalign*} 
    h^{\prime}\left(lfp_{t}\right)&=u^{\prime}\left(c_{t}\right)\left(1-t_{t}\right)w_{t}+u^{\prime}\left(c_{t}\right)b^{ue}_{t}\dfrac{1-p^{f}_{t}}{p^{f}_{t}}-h^{\prime}\left(lfp_{t}\right)\dfrac{1-p^{f}_{t}}{p^{f}_{t}}&\\
    &\quad\quad -\beta\dfrac{1-p^{f}_{t+1}}{p^{f}_{t+1}}\left(1-\rho\right)u^{\prime}\left(c_{t+1}\right)b^{ue}_{t+1}+\beta\dfrac{1-p^{f}_{t+1}}{p^{f}_{t+1}}\left(1-\rho\right)h^{\prime}\left(lfp_{t+1}\right)&
\end{flalign*}

$\iff$
\begin{flalign*} 
    h^{\prime}\left(lfp_{t}\right)+h^{\prime}\left(lfp_{t}\right)\dfrac{1-p^{f}_{t}}{p^{f}_{t}}&=u^{\prime}\left(c_{t}\right)\left(\left(1-t_{t}\right)w_{t}+b^{ue}_{t}\dfrac{1-p^{f}_{t}}{p^{f}_{t}}\right)&\\
    &\quad\quad -\beta\dfrac{1-p^{f}_{t+1}}{p^{f}_{t+1}}\left(1-\rho\right)u^{\prime}\left(c_{t+1}\right)b^{ue}_{t+1}+\beta\dfrac{1-p^{f}_{t+1}}{p^{f}_{t+1}}\left(1-\rho\right)h^{\prime}\left(lfp_{t+1}\right)&
\end{flalign*}

$\iff$
\begin{flalign*} 
    h^{\prime}\left(lfp_{t}\right)\left(1+\dfrac{1-p^{f}_{t}}{p^{f}_{t}}\right)&=u^{\prime}\left(c_{t}\right)\left(\left(1-t_{t}\right)w_{t}+b^{ue}_{t}\dfrac{1-p^{f}_{t}}{p^{f}_{t}}\right)&\\
    &\quad\quad -\beta\dfrac{1-p^{f}_{t+1}}{p^{f}_{t+1}}\left(1-\rho\right)u^{\prime}\left(c_{t+1}\right)b^{ue}_{t+1}+\beta\dfrac{1-p^{f}_{t+1}}{p^{f}_{t+1}}\left(1-\rho\right)h^{\prime}\left(lfp_{t+1}\right)&
\end{flalign*}

$\iff$
\begin{flalign*} 
    h^{\prime}\left(lfp_{t}\right)\left(1+\dfrac{1}{p^{f}_{t}}-1\right)&=u^{\prime}\left(c_{t}\right)\left(\left(1-t_{t}\right)w_{t}+b^{ue}_{t}\dfrac{1-p^{f}_{t}}{p^{f}_{t}}\right)&\\
    &\quad\quad -\beta\dfrac{1-p^{f}_{t+1}}{p^{f}_{t+1}}\left(1-\rho\right)u^{\prime}\left(c_{t+1}\right)b^{ue}_{t+1}+\beta\dfrac{1-p^{f}_{t+1}}{p^{f}_{t+1}}\left(1-\rho\right)h^{\prime}\left(lfp_{t+1}\right)&
\end{flalign*}

$\iff$
\begin{flalign*} 
    h^{\prime}\left(lfp_{t}\right)\dfrac{1}{p^{f}_{t}}&=u^{\prime}\left(c_{t}\right)\left(\left(1-t_{t}\right)w_{t}+b^{ue}_{t}\dfrac{1-p^{f}_{t}}{p^{f}_{t}}\right)&\\
    &\quad\quad -\beta\dfrac{1-p^{f}_{t+1}}{p^{f}_{t+1}}\left(1-\rho\right)u^{\prime}\left(c_{t+1}\right)b^{ue}_{t+1}+\beta\dfrac{1-p^{f}_{t+1}}{p^{f}_{t+1}}\left(1-\rho\right)h^{\prime}\left(lfp_{t+1}\right)&
\end{flalign*}

$\iff$
\begin{flalign*} 
    h^{\prime}\left(lfp_{t}\right)&=u^{\prime}\left(c_{t}\right)\left(p^{f}_{t}\left(1-t_{t}\right)w_{t}+b^{ue}_{t}\left(1-p^{f}_{t}\right)\right)&\\
    &\quad\quad -\beta p^{f}_{t}\dfrac{1-p^{f}_{t+1}}{p^{f}_{t+1}}\left(1-\rho\right)u^{\prime}\left(c_{t+1}\right)b^{ue}_{t+1}+\beta p^{f}_{t}\dfrac{1-p^{f}_{t+1}}{p^{f}_{t+1}}\left(1-\rho\right)h^{\prime}\left(lfp_{t+1}\right)&
\end{flalign*}

$\iff$
\begin{flalign*} 
    \dfrac{h^{\prime}\left(lfp_{t}\right)}{u^{\prime}\left(c_{t}\right)}&=\left(p^{f}_{t}\left(1-t_{t}\right)w_{t}+b^{ue}_{t}\left(1-p^{f}_{t}\right)\right)&\\
    &\quad\quad -\dfrac{\beta p^{f}_{t}\dfrac{1-p^{f}_{t+1}}{p^{f}_{t+1}}\left(1-\rho\right)u^{\prime}\left(c_{t+1}\right)b^{ue}_{t+1}}{u^{\prime}\left(c_{t}\right)}+\dfrac{\beta p^{f}_{t}\dfrac{1-p^{f}_{t+1}}{p^{f}_{t+1}}\left(1-\rho\right)h^{\prime}\left(lfp_{t+1}\right)}{u^{\prime}\left(c_{t}\right)}& 
\end{flalign*}

Assume $b^{ue}_{t}=b^{ue}_{t+1}=b$

$\iff$
\begin{flalign*} \label{eq:1.6}
    \dfrac{h^{\prime}\left(lfp_{t}\right)}{u^{\prime}\left(c_{t}\right)}&=\left(p^{f}_{t}\left(1-t_{t}\right)w_{t}+b\left(1-p^{f}_{t}\right)\right)&\\
    &\quad\quad +p^{f}_{t}\dfrac{1-p^{f}_{t+1}}{p^{f}_{t+1}}\left(1-\rho\right)\left(\dfrac{\beta h^{\prime}\left(lfp_{t+1}\right)}{u^{\prime}\left(c_{t}\right)}-\dfrac{\beta u^{\prime}\left(c_{t+1}\right)b}{u^{\prime}\left(c_{t}\right)}\right)& \tag{1.6}
\end{flalign*}

$\iff$
\begin{flalign*} 
    \dfrac{h^{\prime}\left(lfp_{t}\right)}{u^{\prime}\left(c_{t}\right)}&=\left(p^{f}_{t}\left(1-t_{t}\right)w_{t}+b\left(1-p^{f}_{t}\right)\right)&\\
    &\quad\quad +p^{f}_{t}\dfrac{1-p^{f}_{t+1}}{p^{f}_{t+1}}\left(1-\rho\right)\left(\dfrac{\beta u^{\prime}\left(c_{t+1}\right)}{u^{\prime}\left(c_{t}\right)}\dfrac{h^{\prime}\left(lfp_{t+1}\right)}{u^{\prime}\left(c_{t+1}\right)}-\dfrac{\beta u^{\prime}\left(c_{t+1}\right)b}{u^{\prime}\left(c_{t}\right)}\right)& 
\end{flalign*}

\subsubsection*{d.}

A job is an asset for consumers.

All these relative to the value of unemployment benefit $b$.

The pricing kernel: $\dfrac{\beta u^{\prime}\left(c_{t+1}\right)}{u^{\prime}\left(c_{t}\right)}$, working as a type of asset long lasting relationship. The opportunity cost across periods. If workers found a job in one period, they will not need to search for jobs in the next period if they keep their jobs.

The asset value (of a job) is the avoided disutility from searching a job in period $t+1$: $\dfrac{h^{\prime}\left(lfp_{t+1}\right)}{u^{\prime}\left(c_{t+1}\right)}$.

The adjustment for relative probability of not finding a job next period: $\dfrac{1-p^{f}_{t+1}}{p^{f}_{t+1}}$.

If $p^{f}_{t+1}$ increases, $-\dfrac{h^{\prime}\left(lfp_{t}\right)}{u^{\prime}\left(c_{t}\right)}$ increases (the absolute value decreases.) Higher probability of successfully finding work in the future encourages workers to wait for a period.

If $b$ increases, $-\dfrac{h^{\prime}\left(lfp_{t}\right)}{u^{\prime}\left(c_{t}\right)}$ decreases (usually it is the case). Consumers will be better off and consume more. 

If $\beta$ increases, $-\dfrac{h^{\prime}\left(lfp_{t}\right)}{u^{\prime}\left(c_{t}\right)}$ decreases. Workers will be less impatient, and attach importance to the longer term. It would be easier for workers to find jobs successfully as time goes by.

\subsubsection*{e.}

\fbox{%
	\parbox{1\linewidth}{%
\begin{flalign*} \label{eq:1.7}
    n_{t+1}&=\left(1-\rho\right)n_{t}+v_{t+1}q_{t+1}& \tag{1.7}
\end{flalign*}%
}%
}

$n_{t}$ represents labor demand $n^{d}_{t}$.

The labor in the $t+1$ period is determined by two terms: for the fraction of keeping their job, they stay in the labor market in the next period $(t+1)$. For the vacancy, the firm will find the worker with probability $q$. 

In the classical model we assume that the firm can make the best of their labor demand, and we didn't take the matching into consideration (in other words, firm only consider $n^{d}$ and we suppose the equilibrium $n^{s}=n^{d}$).

Similar to the consumer's problem, the difference between the dynamic model and the static model is that when we introduce $\rho$, there are workers who stay in their job for the next period. Variables could be different in each time period, so we have the subscript.

\subsubsection*{f.}

\fbox{%
	\parbox{1\linewidth}{%
The Lagrangian:
\begin{align*}
    \quad\mathcal{L} & = \sum_{t=0}^{\infty}\left(\dfrac{1}{\displaystyle\prod_{i=0}^{t}\left(1+r_{i}\right)}\left(\left(A_{t}f\left(n_{t}\right)-w_{t}n_{t}-\omega_{t} v_{t}\right)-\mu^{f}_{t}\left(n_{t}-\left(1-\rho\right)n_{t-1}-v_{t}q_{t}\right)\right)\right)&\\
    &=\dots&\\
    &\quad+A_{t}f\left(n_{t}\right)-w_{t}n_{t}-\omega_{t} v_{t}&\\
    &\quad\quad-\mu^{f}_{t}\left(n_{t}-\left(1-\rho\right)n_{t-1}-v_{t}q_{t}\right)&\\
    &\quad+\dfrac{1}{1+r_{t}}A_{t+1}f\left(n_{t+1}\right)-w_{t+1}n_{t+1}-\omega_{t+1} v_{t+1}&\\
    &\quad\quad-\dfrac{1}{1+r_{t}}\mu^{f}_{t+1}\left(n_{t+1}-\left(1-\rho\right)n_{t}-v_{t+1}q_{t+1}\right)&\\
    &\quad+\dots
\end{align*}%
}%
}

The choice variables: $\boxed{n_{t+1}$, $v_{t+1}}$ (in the period $t$).

The given variables: $\boxed{r_{t}$, $A_{t}$, $q_{t}$, $w_{t}$, $\rho$, $\omega}$.

\subsubsection*{g.}

\fbox{%
	\parbox{1\linewidth}{%
    FOCs:

    w.r.t $n_{t}$:
    \begin{flalign} \label{eq:1.8}
        A_{t}f^{\prime}\left(n_{t}\right)-w_{t}-\mu^{f}_{t}+\dfrac{1}{1+r_{t}}\mu^{f}_{t+1}\left(1-\rho\right)&=0& \tag{1.8}
    \end{flalign}

    w.r.t $v_{t}$:
    \begin{flalign*} \label{eq:1.9}
        -\omega_{t}+\mu^{f}_{t}q_{t}&=0& \tag{1.9}
    \end{flalign*}%
}%
}

\eqref{eq:1.9} $\iff$\begin{flalign*} 
    \mu^{f}_{t}q_{t}&=\omega_{t}&
\end{flalign*}

$\iff$
\begin{flalign*} 
    \mu^{f}_{t}&=\dfrac{\omega_{t}}{q_{t}}&
\end{flalign*}

And
\begin{flalign*} 
    \mu^{f}_{t+1}&=\dfrac{\omega_{t+1}}{q_{t+1}}&
\end{flalign*}

Plug into \eqref{eq:1.8}, we get:
\begin{flalign*} 
    A_{t}f^{\prime}\left(n_{t}\right)-w_{t}-\dfrac{\omega_{t}}{q_{t}}+\dfrac{1}{1+r_{t}}\dfrac{\omega_{t+1}}{q_{t+1}}\left(1-\rho\right)&=0& 
\end{flalign*}

\begin{flalign*} \label{eq:1.10}
    \dfrac{\omega_{t}}{q_{t}}&=A_{t}f^{\prime}\left(n_{t}\right)-w_{t}+\dfrac{1}{1+r_{t}}\dfrac{\omega_{t+1}}{q_{t+1}}\left(1-\rho\right)& \tag{1.10}
\end{flalign*}

\subsubsection*{h.}

Vacancy Posting Condition is determined by:

For the workers already hired by the firm, maximize the profit in the factory.

Make the decision of keeping workers and posting new vacancies, across periods.

If $q^{f}_{t+1}$ increases, the marginal benefit of posting a vacancy decreases. Lower probability of successfully hiring a suitable job candidate discourages the firm from posting. 

If $\rho$ increases, the marginal benefit of posting a vacancy decreases. The firm has to post less jobs due to the job retention ``constraint".

If $r_{t}$ increases, the marginal benefit of posting a vacancy decreases. Firms will be more impatient, attach importance to the workers' performance in the short run, thus increasing the vacancy posted once the firm is unsatisfied with workers.

\subsubsection*{i.}

The matching function:
\begin{flalign*} \label{eq:1.11}
    m\left(s_{t},v_{t}\right)&=s^{\gamma}_{t}v^{1-\gamma}_{t}& \tag{1.11}
\end{flalign*}

Market tightness:
\begin{flalign*} \label{eq:1.12}
    \theta_{t}&=\dfrac{v_{t}}{s_{t}}& \tag{1.12}
\end{flalign*}

By definition:
\begin{flalign*} \label{eq:1.13}
    m\left(s_{t},v_{t}\right)&=s_{t}p^{f}_{t}& \tag{1.13}
\end{flalign*}
\begin{flalign*} \label{eq:1.14}
    m\left(s_{t},v_{t}\right)&=v_{t}q_{t}& \tag{1.14}
\end{flalign*}

In equilibrium:
\begin{flalign*} \label{eq:1.15}
    n_{t}&=n^{s}_{t}& \tag{1.15}
\end{flalign*}

Good market clearing condition:
\begin{flalign*} \label{eq:1.16}
    c_{t}&=A_{t}f\left(n_{t}\right)& \tag{1.16}
\end{flalign*}

We also have:
\begin{flalign*} \label{eq:1.17}
    \dfrac{\beta u^{\prime}\left(c_{t+1}\right)}{u^{\prime}\left(c_{t}\right)}&=\dfrac{1}{1+r_{t}}& \tag{1.17}
\end{flalign*}

By \eqref{eq:1.1}, \eqref{eq:1.7} \& \eqref{eq:1.15} we get:

$\left(1-\rho\right)n_{t-1}+p^{f}_{t}s_{t}=\left(1-\rho\right)n_{t-1}+v_{t}q_{t}$

$p^{f}_{t}s_{t}=v_{t}q_{t}\equiv m\left(s_{t},v_{t}\right)$

$\therefore p^{f}_{t}=\dfrac{s^{\gamma}_{t}v^{1-\gamma}_{t}}{s_{t}}=\left(\dfrac{v_{t}}{s_{t}}\right)^{1-\gamma}=\theta^{1-\gamma}_{t}$

$q_{t}=\dfrac{s^{\gamma}_{t}v^{1-\gamma}_{t}}{v_{t}}=\left(\dfrac{s_{t}}{v_{t}}\right)^{\gamma}=\left(\dfrac{v_{t}}{s_{t}}\right)^{-\gamma}=\theta^{-\gamma}_{t}$
\begin{flalign*} \label{eq:1.18}
    p^{f}_{t}&=\theta^{1-\gamma}_{t}& \tag{1.18}
\end{flalign*}
\begin{flalign*} \label{eq:1.19}
    q_{t}&=\theta^{-\gamma}_{t}& \tag{1.19}
\end{flalign*}

We assume
\begin{flalign*} \label{eq:1.20}
    b&=0& \tag{1.20}
\end{flalign*}
\begin{flalign*} \label{eq:1.21}
    t_{t}&=0& \tag{1.21}
\end{flalign*}

Plug \eqref{eq:1.20} \& \eqref{eq:1.21} into \eqref{eq:1.2} we have:
\begin{flalign*} \label{eq:1.22}
    c_{t}&=w_{t}n^{s}_{t}& \tag{1.22}
\end{flalign*}

Plug \eqref{eq:1.20} \& \eqref{eq:1.21} into \eqref{eq:1.6} we have:
\begin{flalign*} \label{eq:1.23}
    \dfrac{h^{\prime}\left(lfp_{t}\right)}{u^{\prime}\left(c_{t}\right)}&=p^{f}_{t}w_{t}+\dfrac{\beta p^{f}_{t}\dfrac{1-p^{f}_{t+1}}{p^{f}_{t+1}}\left(1-\rho\right)h^{\prime}\left(lfp_{t+1}\right)}{u^{\prime}\left(c_{t}\right)}& \tag{1.23}
\end{flalign*}

By \eqref{eq:1.18} \& \eqref{eq:1.23}:
\begin{flalign*} \label{eq:1.24}
    \dfrac{h^{\prime}\left(lfp_{t}\right)}{u^{\prime}\left(c_{t}\right)}&=\theta^{1-\gamma}_{t}w_{t}+\dfrac{\beta \theta^{1-\gamma}_{t}\dfrac{1-\theta^{1-\gamma}_{t+1}}{\theta^{1-\gamma}_{t+1}}\left(1-\rho\right)h^{\prime}\left(lfp_{t+1}\right)}{u^{\prime}\left(c_{t}\right)}& \tag{1.24}
\end{flalign*}

By \eqref{eq:1.10} \& \eqref{eq:1.19} \& \eqref{eq:1.17}:
\begin{flalign*} \label{eq:1.25}
    \dfrac{\omega_{t}}{\theta^{-\gamma}_{t}}&=A_{t}f^{\prime}\left(n_{t}\right)-w_{t}+\dfrac{\beta u^{\prime}\left(c_{t+1}\right)}{u^{\prime}\left(c_{t}\right)}\dfrac{\omega_{t+1}}{\theta^{-\gamma}_{t+1}}\left(1-\rho\right)& \tag{1.25}
\end{flalign*}

By \eqref{eq:1.16} \& \eqref{eq:1.22}:
\begin{flalign*} 
    A_{t}f\left(n_{t}\right)&=w_{t}n^{s}_{t}& 
\end{flalign*}
\begin{flalign*} 
    \therefore A_{t}f^{\prime}\left(n_{t}\right)&=\dfrac{\partial w_{t}}{\partial n_{t}}n_{t}+w_{t}& 
\end{flalign*}
\begin{flalign*} \label{eq:1.26}
    \dfrac{\omega_{t}}{\theta^{-\gamma}_{t}}&=A_{t}f^{\prime}\left(n_{t}\right)-w_{t}+\dfrac{\beta u^{\prime}\left(c_{t+1}\right)}{u^{\prime}\left(c_{t}\right)}\dfrac{\omega_{t+1}}{\theta^{-\gamma}_{t+1}}\left(1-\rho\right)& \tag{1.26}
\end{flalign*}

With \eqref{eq:1.15}, \eqref{eq:1.22}, \eqref{eq:1.24} \& \eqref{eq:1.26}:

\fbox{%
	\parbox{1\linewidth}{%
\begin{flalign*} 
    \dfrac{h^{\prime}\left(lfp_{t}\right)}{u^{\prime}\left(w_{t}n_{t}\right)}&=\theta^{1-\gamma}_{t}w_{t}+\dfrac{\beta \theta^{1-\gamma}_{t}\dfrac{1-\theta^{1-\gamma}_{t+1}}{\theta^{1-\gamma}_{t+1}}\left(1-\rho\right)h^{\prime}\left(lfp_{t+1}\right)}{u^{\prime}\left(w_{t}n_{t}\right)}& 
\end{flalign*}
\begin{flalign*} 
    \dfrac{\omega_{t}}{\theta^{-\gamma}_{t}}&=A_{t}f^{\prime}\left(n_{t}\right)-w_{t}+\dfrac{\beta u^{\prime}\left(w_{t+1}n_{t+1}\right)}{u^{\prime}\left(w_{t}n_{t}\right)}\dfrac{\omega_{t+1}}{\theta^{-\gamma}_{t+1}}\left(1-\rho\right)& 
\end{flalign*}%
}%
}

\subsection*{2. Optimal Policy in the NK model}

\subsubsection*{a.}

Solving consumer's problem $\implies$ Solving retail firm's problem $\implies$ Solving wholesale firm's problem and get NKPC $\implies$ Introducing government's problem $\implies$ Summarize equilibrium conditions and substitute, solving system of equation $\implies$ Imposing steady state $\implies$ Maximizing government's goal $\implies$ Modifying conditions to get optimal policy.

\subsubsection*{b.}

Consumer's problem:

\fbox{%
	\parbox{1\linewidth}{%
\begin{equation*}
    \begin{aligned}
    & \quad\max\sum_{s=0}^{\infty}\beta^{s}u\left(c_{t},1-n_{t}\right)&\\
     \textrm{subject to}&\\
    & P_{t}c_{t}+P^{b}_{t}B_{t}+S_{t}a_{t}=P_{t}w_{t}n_{t}+B_{t-1}+\left(S_{t}+D_{t}\right)a_{t-1}.
    \end{aligned}
\end{equation*}%
}%
}

The Lagrangian:
\begin{align*}
    \quad\mathcal{L} & = \sum_{s=0}^{\infty}\beta^{s}u\left(c_{t},1-n_{t}\right)& \\
    &\quad\quad-\lambda_{t}\left(P_{t}c_{t}+P^{b}_{t}B_{t}+S_{t}a_{t}-P_{t}w_{t}n_{t}-B_{t-1}-\left(S_{t}+D_{t}\right)a_{t-1}\right)&\\
    &\quad\quad-\beta\lambda_{t+1}\left(P_{t+1}c_{t+1}+P^{b}_{t+1}B_{t+1}+S_{t+1}a_{t+1}-P_{t+1}w_{t+1}n_{t+1}-B_{t}-\left(S_{t+1}+D_{t+1}\right)a_{t}\right)&\\
    &\quad\quad+\dots
\end{align*}

\fbox{%
	\parbox{1\linewidth}{%
FOCs

w.r.t $c_{t}$:
\begin{flalign} \label{eq:2.1}
    u_{1}\left(c_{t},1-n_{t}\right)-\lambda_{t}P_{t}&=0& \tag{2.1}
\end{flalign}

w.r.t $n_{t}$:
\begin{flalign} \label{eq:2.2}
    u_{2}\left(c_{t},1-n_{t}\right)+\lambda_{t}P_{t}w_{t}&=0& \tag{2.2}
\end{flalign}

where $u_{2}=u_{l}=-u_{n}$

w.r.t $B_{t}$:
\begin{flalign} \label{eq:2.3}
    -\lambda_{t}P^{b}_{t}+\beta\lambda_{t+1}&=0& \tag{2.3}
\end{flalign}

w.r.t $a_{t}$:
\begin{flalign} \label{eq:2.4}
    -\lambda_{t}S_{t}+\beta\lambda_{t+1}\left(S_{t+1}+D_{t+1}\right)&=0& \tag{2.4}
\end{flalign}

w.r.t $\lambda_{t}$:
\begin{flalign} \label{eq:2.5}
    P_{t}c_{t}+P^{b}_{t}B_{t}+S_{t}a_{t}&=P_{t}w_{t}n_{t}+B_{t-1}+\left(S_{t}+D_{t}\right)a_{t-1}& \tag{2.5}
\end{flalign}%
}%
}

\subsubsection*{c.}

\fbox{%
	\parbox{1\linewidth}{%
Resource constraint:
\begin{flalign} \label{eq:2.6}
    c_{t}+\dfrac{\psi}{2}\left(\pi_{t}\right)^{2}&=n_{t}& \tag{2.6}
\end{flalign}%
}%
}

Since we assume CRS production function

$\quad y_{t}=n_{t}$,

resource constraint implies that the market clears. Otherwise, resources haven't been used or overused, and both consumer and firm can make the best of this situation and become better.

\subsubsection*{d.}

Divide \eqref{eq:2.2} with \eqref{eq:2.1} we have:

\fbox{%
	\parbox{1\linewidth}{%
\begin{flalign} \label{eq:2.7}
    \dfrac{u_{2}\left(c_{t},1-n_{t}\right)}{u_{1}\left(c_{t},1-n_{t}\right)}&=w_{t}& \tag{2.7}
\end{flalign}%
}%
}

By \eqref{eq:2.1} we have:

\begin{flalign} \label{eq:2.8}
    u_{1}\left(c_{t},1-n_{t}\right)&=\lambda_{t}P_{t}& \tag{2.8}
\end{flalign}

\begin{flalign} \label{eq:2.9}
    u_{1}\left(c_{t+1},1-n_{t+1}\right)&=\lambda_{t+1}P_{t+1}& \tag{2.9}
\end{flalign}

Divide \eqref{eq:2.8} with \eqref{eq:2.9} we have:
\begin{flalign} \label{eq:2.10}
    \dfrac{u_{1}\left(c_{t},1-n_{t}\right)}{u_{1}\left(c_{t+1},1-n_{t+1}\right)}&=\dfrac{\lambda_{t}P_{t}}{\lambda_{t+1}P_{t+1}}& \tag{2.10}
\end{flalign}

By the definition of the inflation, we have:
\begin{flalign} \label{eq:2.11}
    1+\pi_{t+1}&=\dfrac{P_{t+1}}{P_{t}}& \tag{2.11}
\end{flalign}

By \eqref{eq:2.10} and \eqref{eq:2.11} we get:
\begin{flalign} \label{eq:2.12}
    \dfrac{u_{1}\left(c_{t},1-n_{t}\right)}{u_{1}\left(c_{t+1},1-n_{t+1}\right)}&=\dfrac{\lambda_{t}}{\lambda_{t+1}}\cdot\dfrac{1}{\left(1+\pi_{t+1}\right)}& \tag{2.12}
\end{flalign}

By \eqref{eq:2.3} we have:

$\quad \lambda_{t}P^{b}_{t}=\beta\lambda_{t+1}$

$\iff$

$\quad \lambda_{t}=\dfrac{\beta\lambda_{t+1}}{P^{b}_{t}}$
\begin{flalign} \label{eq:2.13}
    \dfrac{1}{P^{b}_{t}}&=\dfrac{\lambda_{t}}{\beta\lambda_{t+1}}& \tag{2.13}
\end{flalign}

By the definition of the interest rate, we have:
\begin{flalign} \label{eq:2.14}
    \dfrac{1}{P^{b}_{t}}&=1+i_{t}& \tag{2.14}
\end{flalign}

Plug in \eqref{eq:2.13} we get:
\begin{flalign*} 
    \dfrac{\lambda_{t}}{\beta\lambda_{t+1}}&=1+i_{t}& 
\end{flalign*}

Plug this equation back to the \eqref{eq:2.12} we get:

\fbox{%
	\parbox{1\linewidth}{%
\begin{flalign} \label{eq:2.15}
    \dfrac{u_{1}\left(c_{t},1-n_{t}\right)}{\beta u_{1}\left(c_{t+1},1-n_{t+1}\right)}&=\dfrac{1+i_{t}}{1+\pi_{t+1}}& \tag{2.15}
\end{flalign}%
}%
}

We also have:
\begin{flalign} \label{eq:2.16}
    mc_{t}&=w_{t}& \tag{2.16}
\end{flalign}

Because $mp_{t}=f^{\prime}\left(n_{t}\right)=\dfrac{\partial n_{t}}{\partial n_{t}}=1>mc_{t}$

Labor demand condition:

\fbox{%
	\parbox{1\linewidth}{%
\begin{flalign} \label{eq:2.17}
    w_{t}&<1& \tag{2.17}
\end{flalign}%
}%
}

By the previous chapter and A5, we have the NKPC:

\fbox{%
	\parbox{1\linewidth}{%
\begin{flalign} \label{eq:2.18}
    \dfrac{1}{1-\varepsilon}\left(1-\varepsilon mc_{t}\right)n_{t}-\psi\pi_{t}\left(1+\pi_{t}\right)+\beta\psi\pi_{t+1}\left(1+\pi_{t+1}\right)&=0& \tag{2.18}
\end{flalign}%
}%
}

And the resource constraint \eqref{eq:2.6}:

\fbox{%
	\parbox{1\linewidth}{%
\begin{flalign*} 
    c_{t}+\dfrac{\psi}{2}\left(\pi_{t}\right)^{2}&=n_{t}& 
\end{flalign*}%
}%
}

\subsubsection*{e.}

From \eqref{eq:2.6} \& \eqref{eq:2.7} \& \eqref{eq:2.16} we get:
\begin{flalign} \label{eq:2.19}
    \dfrac{u_{2}\left(c_{t},1-c_{t}-\dfrac{\psi}{2}\left(\pi_{t}\right)^{2}\right)}{u_{1}\left(c_{t},1-c_{t}-\dfrac{\psi}{2}\left(\pi_{t}\right)^{2}\right)}&=mc_{t}& \tag{2.19}
\end{flalign}

Substitute \eqref{eq:2.19} into \eqref{eq:2.15} we have:
\begin{flalign*} \label{eq:2.20}
    \dfrac{u_{1}\left(c_{t},1-c_{t}-\dfrac{\psi}{2}\left(\pi_{t}\right)^{2}\right)}{\beta u_{1}\left(c_{t+1},1-c_{t+1}-\dfrac{\psi}{2}\left(\pi_{t+1}\right)^{2}\right)}&=\dfrac{1+i_{t}}{1+\pi_{t+1}}& \tag{2.20}
\end{flalign*}

Substitute \eqref{eq:2.19} into \eqref{eq:2.18} we have:
\begin{flalign*} \label{eq:2.21}
    \dfrac{1}{1-\varepsilon}\left(1-\dfrac{\varepsilon u_{2}\left(c_{t},1-c_{t}-\dfrac{\psi}{2}\pi_{t}^{2}\right)}{u_{1}\left(c_{t},1-c_{t}-\dfrac{\psi}{2}\pi_{t}^{2}\right)}\right)\left(c_{t}+\dfrac{\psi}{2}\pi_{t}^{2}\right)-\psi\pi_{t}\left(1+\pi_{t}\right)+\beta\psi\pi_{t+1}\left(1+\pi_{t+1}\right)&=0& \tag{2.21}
\end{flalign*}

In the steady state, rewriting \eqref{eq:2.19} \& \eqref{eq:2.20} \& \eqref{eq:2.21}:

\fbox{%
	\parbox{1\linewidth}{%
\begin{flalign} \label{eq:2.22}
    \dfrac{u_{2}\left(c,1-c-\dfrac{\psi}{2}\pi^{2}\right)}{u_{1}\left(c,1-c-\dfrac{\psi}{2}\pi^{2}\right)}&=w& \tag{2.22}
\end{flalign}%
}%
}

\fbox{%
	\parbox{1\linewidth}{%
\begin{flalign*} \label{eq:2.23}
    \dfrac{u_{1}\left(c,1-c-\dfrac{\psi}{2}\pi^{2}\right)}{\beta u_{1}\left(c,1-c-\dfrac{\psi}{2}\pi^{2}\right)}&=\dfrac{1+i}{1+\pi}& \tag{2.23}
\end{flalign*}%
}%
}

\begin{flalign*} \label{eq:2.24}
    \dfrac{1}{1-\varepsilon}\left(1-\dfrac{\varepsilon u_{2}\left(c,1-c-\dfrac{\psi}{2}\pi^{2}\right)}{u_{1}\left(c,1-c-\dfrac{\psi}{2}\pi^{2}\right)}\right)\left(c+\dfrac{\psi}{2}\pi^{2}\right)-\psi\pi\left(1+\pi\right)+\beta\psi\pi\left(1+\pi\right)&=0& \tag{2.24}
\end{flalign*}

Also, 
\begin{flalign} \label{eq:2.25}
    g&=\pi& \tag{2.25}
\end{flalign}
in a steady state. Plug into \eqref{eq:2.24}:

\fbox{%
	\parbox{1\linewidth}{%
\begin{flalign*} \label{eq:2.26}
    \dfrac{1}{1-\varepsilon}\left(1-\dfrac{\varepsilon u_{2}\left(c,1-c-\dfrac{\psi}{2}g^{2}\right)}{u_{1}\left(c,1-c-\dfrac{\psi}{2}g^{2}\right)}\right)\left(c+\dfrac{\psi}{2}g^{2}\right)-\psi g\left(1+\pi\right)+\beta\psi g\left(1+g\right)&=0& \tag{2.26}
\end{flalign*}%
}%
}

\subsubsection*{f.}

Optimal monetary policy problem:

\fbox{%
	\parbox{1\linewidth}{%
\begin{equation*}
    \begin{aligned}
    & \quad\max_{g}\sum_{s=0}^{\infty}\beta^{s}u\left(\bar{c}\left(g\right),1-\bar{c}\left(g\right)-\dfrac{\psi}{2}g^{2}\right)
    \end{aligned}
\end{equation*}%
}%
}

$\iff$
\begin{equation*}
    \begin{aligned}
    & \quad\max_{g} \dfrac{u\left(\bar{c}\left(g\right),1-\bar{c}\left(g\right)-\dfrac{\psi}{2}g^{2}\right)}{1-\beta}
    \end{aligned}
\end{equation*}

The policy tries to maximize the utility of the society. In the representative case, it is the same with maximizing consumer's utility.

The choice variable is $g$.

\subsubsection*{g.}

FOC w.r.t $g$:

$\quad u_{1}\dfrac{\partial \bar{c}\left(g\right)}{\partial g}+u_{2}\cdot\dfrac{\partial \left(1-\bar{c}\left(g\right)-\dfrac{\psi}{2}g^{2}\right)}{\partial g}=0$

$\iff$

$\quad u_{1}c^{\prime}\left(g\right)+u_{2}\left(-c^{\prime}\left(g\right)-\psi g\right)=0$

$\iff$

$\quad u_{1}c^{\prime}\left(g\right)=u_{2}\left(c^{\prime}\left(g\right)+\psi g\right)$

$\iff$

$\quad \dfrac{u_{2}\left(c\left(g\right), 1-c\left(g\right)-\dfrac{\psi}{2}g^{2}\right)}{u_{1}\left(c\left(g\right), 1-c\left(g\right)-\dfrac{\psi}{2}g^{2}\right)}=\dfrac{c^{\prime}\left(g\right)}{c^{\prime}\left(g\right)+\psi g}$

\subsubsection*{h.}

Introduce $\varepsilon$ in the steady state NKPC \eqref{eq:2.26}:
\begin{flalign*} 
    \dfrac{1}{1-\varepsilon}\left(1-\dfrac{\varepsilon u_{2}\left(c,1-c-\dfrac{\psi}{2}g^{2}\right)}{\boxed{\varepsilon} u_{1}\left(c,1-c-\dfrac{\psi}{2}g^{2}\right)}\right)\left(c+\dfrac{\psi}{2}g^{2}\right)-\psi g\left(1+\pi\right)+\beta\psi g\left(1+g\right)&=0& 
\end{flalign*}

And we assume $\psi=0$.

By inspection, $\dfrac{u_{2}\left(c\left(g\right), 1-c\left(g\right)\right)}{u_{1}\left(c\left(g\right), 1-c\left(g\right)\right)}=1$ solves this problem. That is, $g^{*}=0$.

Zero inflation eliminates the menu cost, which has no benefit. Thus there is no need to change prices in this model, and the private market achieves efficiency. 

\end{document}